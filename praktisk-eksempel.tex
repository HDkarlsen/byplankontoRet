% Options for packages loaded elsewhere
\PassOptionsToPackage{unicode}{hyperref}
\PassOptionsToPackage{hyphens}{url}
\PassOptionsToPackage{dvipsnames,svgnames,x11names}{xcolor}
%
\documentclass[
  letterpaper,
  DIV=11,
  numbers=noendperiod]{scrartcl}

\usepackage{amsmath,amssymb}
\usepackage{lmodern}
\usepackage{iftex}
\ifPDFTeX
  \usepackage[T1]{fontenc}
  \usepackage[utf8]{inputenc}
  \usepackage{textcomp} % provide euro and other symbols
\else % if luatex or xetex
  \usepackage{unicode-math}
  \defaultfontfeatures{Scale=MatchLowercase}
  \defaultfontfeatures[\rmfamily]{Ligatures=TeX,Scale=1}
\fi
% Use upquote if available, for straight quotes in verbatim environments
\IfFileExists{upquote.sty}{\usepackage{upquote}}{}
\IfFileExists{microtype.sty}{% use microtype if available
  \usepackage[]{microtype}
  \UseMicrotypeSet[protrusion]{basicmath} % disable protrusion for tt fonts
}{}
\makeatletter
\@ifundefined{KOMAClassName}{% if non-KOMA class
  \IfFileExists{parskip.sty}{%
    \usepackage{parskip}
  }{% else
    \setlength{\parindent}{0pt}
    \setlength{\parskip}{6pt plus 2pt minus 1pt}}
}{% if KOMA class
  \KOMAoptions{parskip=half}}
\makeatother
\usepackage{xcolor}
\setlength{\emergencystretch}{3em} % prevent overfull lines
\setcounter{secnumdepth}{-\maxdimen} % remove section numbering
% Make \paragraph and \subparagraph free-standing
\ifx\paragraph\undefined\else
  \let\oldparagraph\paragraph
  \renewcommand{\paragraph}[1]{\oldparagraph{#1}\mbox{}}
\fi
\ifx\subparagraph\undefined\else
  \let\oldsubparagraph\subparagraph
  \renewcommand{\subparagraph}[1]{\oldsubparagraph{#1}\mbox{}}
\fi

\usepackage{color}
\usepackage{fancyvrb}
\newcommand{\VerbBar}{|}
\newcommand{\VERB}{\Verb[commandchars=\\\{\}]}
\DefineVerbatimEnvironment{Highlighting}{Verbatim}{commandchars=\\\{\}}
% Add ',fontsize=\small' for more characters per line
\usepackage{framed}
\definecolor{shadecolor}{RGB}{241,243,245}
\newenvironment{Shaded}{\begin{snugshade}}{\end{snugshade}}
\newcommand{\AlertTok}[1]{\textcolor[rgb]{0.68,0.00,0.00}{#1}}
\newcommand{\AnnotationTok}[1]{\textcolor[rgb]{0.37,0.37,0.37}{#1}}
\newcommand{\AttributeTok}[1]{\textcolor[rgb]{0.40,0.45,0.13}{#1}}
\newcommand{\BaseNTok}[1]{\textcolor[rgb]{0.68,0.00,0.00}{#1}}
\newcommand{\BuiltInTok}[1]{\textcolor[rgb]{0.00,0.23,0.31}{#1}}
\newcommand{\CharTok}[1]{\textcolor[rgb]{0.13,0.47,0.30}{#1}}
\newcommand{\CommentTok}[1]{\textcolor[rgb]{0.37,0.37,0.37}{#1}}
\newcommand{\CommentVarTok}[1]{\textcolor[rgb]{0.37,0.37,0.37}{\textit{#1}}}
\newcommand{\ConstantTok}[1]{\textcolor[rgb]{0.56,0.35,0.01}{#1}}
\newcommand{\ControlFlowTok}[1]{\textcolor[rgb]{0.00,0.23,0.31}{#1}}
\newcommand{\DataTypeTok}[1]{\textcolor[rgb]{0.68,0.00,0.00}{#1}}
\newcommand{\DecValTok}[1]{\textcolor[rgb]{0.68,0.00,0.00}{#1}}
\newcommand{\DocumentationTok}[1]{\textcolor[rgb]{0.37,0.37,0.37}{\textit{#1}}}
\newcommand{\ErrorTok}[1]{\textcolor[rgb]{0.68,0.00,0.00}{#1}}
\newcommand{\ExtensionTok}[1]{\textcolor[rgb]{0.00,0.23,0.31}{#1}}
\newcommand{\FloatTok}[1]{\textcolor[rgb]{0.68,0.00,0.00}{#1}}
\newcommand{\FunctionTok}[1]{\textcolor[rgb]{0.28,0.35,0.67}{#1}}
\newcommand{\ImportTok}[1]{\textcolor[rgb]{0.00,0.46,0.62}{#1}}
\newcommand{\InformationTok}[1]{\textcolor[rgb]{0.37,0.37,0.37}{#1}}
\newcommand{\KeywordTok}[1]{\textcolor[rgb]{0.00,0.23,0.31}{#1}}
\newcommand{\NormalTok}[1]{\textcolor[rgb]{0.00,0.23,0.31}{#1}}
\newcommand{\OperatorTok}[1]{\textcolor[rgb]{0.37,0.37,0.37}{#1}}
\newcommand{\OtherTok}[1]{\textcolor[rgb]{0.00,0.23,0.31}{#1}}
\newcommand{\PreprocessorTok}[1]{\textcolor[rgb]{0.68,0.00,0.00}{#1}}
\newcommand{\RegionMarkerTok}[1]{\textcolor[rgb]{0.00,0.23,0.31}{#1}}
\newcommand{\SpecialCharTok}[1]{\textcolor[rgb]{0.37,0.37,0.37}{#1}}
\newcommand{\SpecialStringTok}[1]{\textcolor[rgb]{0.13,0.47,0.30}{#1}}
\newcommand{\StringTok}[1]{\textcolor[rgb]{0.13,0.47,0.30}{#1}}
\newcommand{\VariableTok}[1]{\textcolor[rgb]{0.07,0.07,0.07}{#1}}
\newcommand{\VerbatimStringTok}[1]{\textcolor[rgb]{0.13,0.47,0.30}{#1}}
\newcommand{\WarningTok}[1]{\textcolor[rgb]{0.37,0.37,0.37}{\textit{#1}}}

\providecommand{\tightlist}{%
  \setlength{\itemsep}{0pt}\setlength{\parskip}{0pt}}\usepackage{longtable,booktabs,array}
\usepackage{calc} % for calculating minipage widths
% Correct order of tables after \paragraph or \subparagraph
\usepackage{etoolbox}
\makeatletter
\patchcmd\longtable{\par}{\if@noskipsec\mbox{}\fi\par}{}{}
\makeatother
% Allow footnotes in longtable head/foot
\IfFileExists{footnotehyper.sty}{\usepackage{footnotehyper}}{\usepackage{footnote}}
\makesavenoteenv{longtable}
\usepackage{graphicx}
\makeatletter
\def\maxwidth{\ifdim\Gin@nat@width>\linewidth\linewidth\else\Gin@nat@width\fi}
\def\maxheight{\ifdim\Gin@nat@height>\textheight\textheight\else\Gin@nat@height\fi}
\makeatother
% Scale images if necessary, so that they will not overflow the page
% margins by default, and it is still possible to overwrite the defaults
% using explicit options in \includegraphics[width, height, ...]{}
\setkeys{Gin}{width=\maxwidth,height=\maxheight,keepaspectratio}
% Set default figure placement to htbp
\makeatletter
\def\fps@figure{htbp}
\makeatother

\KOMAoption{captions}{tableheading}
\makeatletter
\makeatother
\makeatletter
\makeatother
\makeatletter
\@ifpackageloaded{caption}{}{\usepackage{caption}}
\AtBeginDocument{%
\ifdefined\contentsname
  \renewcommand*\contentsname{Table of contents}
\else
  \newcommand\contentsname{Table of contents}
\fi
\ifdefined\listfigurename
  \renewcommand*\listfigurename{List of Figures}
\else
  \newcommand\listfigurename{List of Figures}
\fi
\ifdefined\listtablename
  \renewcommand*\listtablename{List of Tables}
\else
  \newcommand\listtablename{List of Tables}
\fi
\ifdefined\figurename
  \renewcommand*\figurename{Figure}
\else
  \newcommand\figurename{Figure}
\fi
\ifdefined\tablename
  \renewcommand*\tablename{Table}
\else
  \newcommand\tablename{Table}
\fi
}
\@ifpackageloaded{float}{}{\usepackage{float}}
\floatstyle{ruled}
\@ifundefined{c@chapter}{\newfloat{codelisting}{h}{lop}}{\newfloat{codelisting}{h}{lop}[chapter]}
\floatname{codelisting}{Listing}
\newcommand*\listoflistings{\listof{codelisting}{List of Listings}}
\makeatother
\makeatletter
\@ifpackageloaded{caption}{}{\usepackage{caption}}
\@ifpackageloaded{subcaption}{}{\usepackage{subcaption}}
\makeatother
\makeatletter
\@ifpackageloaded{tcolorbox}{}{\usepackage[many]{tcolorbox}}
\makeatother
\makeatletter
\@ifundefined{shadecolor}{\definecolor{shadecolor}{rgb}{.97, .97, .97}}
\makeatother
\makeatletter
\makeatother
\ifLuaTeX
  \usepackage{selnolig}  % disable illegal ligatures
\fi
\IfFileExists{bookmark.sty}{\usepackage{bookmark}}{\usepackage{hyperref}}
\IfFileExists{xurl.sty}{\usepackage{xurl}}{} % add URL line breaks if available
\urlstyle{same} % disable monospaced font for URLs
\hypersetup{
  colorlinks=true,
  linkcolor={blue},
  filecolor={Maroon},
  citecolor={Blue},
  urlcolor={Blue},
  pdfcreator={LaTeX via pandoc}}

\author{}
\date{}

\begin{document}
\ifdefined\Shaded\renewenvironment{Shaded}{\begin{tcolorbox}[interior hidden, breakable, frame hidden, boxrule=0pt, sharp corners, enhanced, borderline west={3pt}{0pt}{shadecolor}]}{\end{tcolorbox}}\fi

\hypertarget{et-praktisk-eksempel}{%
\section{Et praktisk eksempel}\label{et-praktisk-eksempel}}

En ting er å lese om R, men en annen ting er å gjøre det i praksis. La
oss jobbe gjennom et konkret, enkelt eksempel og kommentert grundig hva
vi gjorde slik at den som kommer etter oss (våre kolleger eller oss sjøl
i framtida) forstår hva vi gjør.

\hypertarget{oppgavebeskrivelse}{%
\subsection{Oppgavebeskrivelse}\label{oppgavebeskrivelse}}

Vi vil laste inn to SPSS-datasett. Det ene ligger i en undermappe
(relativt til skriptet) kalt \emph{data}. Det andre ligger godt hjem
langt, langt vekk et sted på \emph{C}-disken min. Vi vil laste inn begge
to og se hvordan det de ser ut.

\begin{Shaded}
\begin{Highlighting}[]
\CommentTok{\# For å laste inn SPSS{-}filer trenger vi en pakke som gjør det. Haven er bra.}
\CommentTok{\# Dersom du ikke har haven fra før må den installeres. En pakke trenger bare }
\CommentTok{\# installeres  én gang. Siden jeg har pakka fra før har}
\CommentTok{\# jeg kommentert vekk neste kode for. Skal du installere koden, fjern }
\CommentTok{\# emneknaggen.}

\CommentTok{\# install.packages("haven")}

\CommentTok{\# Så må vi laste inn pakka for å kunne ta dens funksjoner i bruk. Dette må vi}
\CommentTok{\# gjøre hver gang vi starter en ny sesjon.}

\FunctionTok{library}\NormalTok{(haven)}

\CommentTok{\# Vi laster inn den første SPSS{-}fila. Vi gir den navnet atferd. Siden den }
\CommentTok{\# ligger på mappa data må vi spesifisere dette når vi oppgir hvor den ligger og}
\CommentTok{\# hva den heter. Husk også filendelsen. Noen operativsystem er glad i skjule }
\CommentTok{\# fil{-}endelsen, men den er en viktig del av alle filers navn.}
\NormalTok{atferd }\OtherTok{\textless{}{-}} \FunctionTok{read\_sav}\NormalTok{(}\AttributeTok{file =} \StringTok{"data/behavior.sav"}\NormalTok{)}

\CommentTok{\# Den andre fila ligger langt vekk på C. For å gjøre det litt enklere for meg}
\CommentTok{\# sjøl vil jeg lagre filstien (og navnet) i en vektor. Jeg gjør ofte dette om }
\CommentTok{\# filstien er lang, for å ikke gjøre import{-}funksjonen så lang.}
\CommentTok{\# Husk dette med skråstreker: Vi må enten bruke \textbackslash{}\textbackslash{} eller /, ikke en enkelt \textbackslash{}}
\NormalTok{filsti }\OtherTok{\textless{}{-}} \StringTok{"C:}\SpecialCharTok{\textbackslash{}\textbackslash{}}\StringTok{Program Files}\SpecialCharTok{\textbackslash{}\textbackslash{}}\StringTok{IBM}\SpecialCharTok{\textbackslash{}\textbackslash{}}\StringTok{SPSS}\SpecialCharTok{\textbackslash{}\textbackslash{}}\StringTok{Statistics}\SpecialCharTok{\textbackslash{}\textbackslash{}}\StringTok{26}\SpecialCharTok{\textbackslash{}\textbackslash{}}\StringTok{Samples}\SpecialCharTok{\textbackslash{}\textbackslash{}}\StringTok{English}\SpecialCharTok{\textbackslash{}\textbackslash{}}\StringTok{accidents.sav"}

\CommentTok{\# Så laster vi inn den andre fila. }
\NormalTok{ulykker }\OtherTok{\textless{}{-}} \FunctionTok{read\_sav}\NormalTok{(}\AttributeTok{file =}\NormalTok{ filsti)}

\CommentTok{\# Dette er altså det samme som å skrive }
\CommentTok{\# ulykker \textless{}{-} read\_sav(file = "C:\textbackslash{}\textbackslash{}Program Files\textbackslash{}\textbackslash{}IBM\textbackslash{}\textbackslash{}SPSS\textbackslash{}\textbackslash{}Statistics\textbackslash{}\textbackslash{}26\textbackslash{}\textbackslash{}Samples\textbackslash{}\textbackslash{}English\textbackslash{}\textbackslash{}accidents.sav")}
\CommentTok{\# Siden vi definerte filsti tidligere. }

\CommentTok{\# Nå har vi fått de to datasetta våre. Hvordan ser de ut?}

\FunctionTok{summary}\NormalTok{(atferd)}
\end{Highlighting}
\end{Shaded}

\begin{verbatim}
     ROWID           Run             Talk            Kiss           Write      
 Min.   : 1.0   Min.   :1.060   Min.   :0.420   Min.   :0.270   Min.   :0.710  
 1st Qu.: 4.5   1st Qu.:4.300   1st Qu.:0.560   1st Qu.:2.625   1st Qu.:2.810  
 Median : 8.0   Median :6.440   Median :0.920   Median :4.080   Median :4.440  
 Mean   : 8.0   Mean   :5.539   Mean   :1.523   Mean   :3.996   Mean   :4.166  
 3rd Qu.:11.5   3rd Qu.:7.050   3rd Qu.:1.675   3rd Qu.:5.460   3rd Qu.:5.790  
 Max.   :15.0   Max.   :7.620   Max.   :5.710   Max.   :7.920   Max.   :6.420  
      Eat            Sleep           Mumble           Read      
 Min.   :0.560   Min.   :0.150   Min.   :1.330   Min.   :0.420  
 1st Qu.:1.135   1st Qu.:4.145   1st Qu.:3.685   1st Qu.:1.780  
 Median :1.810   Median :6.020   Median :3.960   Median :4.290  
 Mean   :3.148   Mean   :5.311   Mean   :4.431   Mean   :3.905  
 3rd Qu.:4.470   3rd Qu.:6.970   3rd Qu.:5.430   3rd Qu.:5.365  
 Max.   :7.620   Max.   :8.250   Max.   :7.690   Max.   :7.270  
     Fight           Belch           Argue            Jump      
 Min.   :4.750   Min.   :2.190   Min.   :1.480   Min.   :1.580  
 1st Qu.:6.780   1st Qu.:4.500   1st Qu.:4.070   1st Qu.:4.500  
 Median :7.330   Median :6.420   Median :4.830   Median :5.460  
 Mean   :7.035   Mean   :5.731   Mean   :5.027   Mean   :5.265  
 3rd Qu.:7.585   3rd Qu.:6.810   3rd Qu.:6.085   3rd Qu.:6.795  
 Max.   :8.380   Max.   :7.790   Max.   :7.290   Max.   :7.520  
      Cry            Laugh           Shout      
 Min.   :1.000   Min.   :0.770   Min.   :1.060  
 1st Qu.:4.450   1st Qu.:1.030   1st Qu.:4.495  
 Median :5.520   Median :1.600   Median :5.480  
 Mean   :4.999   Mean   :1.993   Mean   :5.317  
 3rd Qu.:5.895   3rd Qu.:2.500   3rd Qu.:7.050  
 Max.   :7.630   Max.   :6.400   Max.   :7.670  
\end{verbatim}

\begin{Shaded}
\begin{Highlighting}[]
\FunctionTok{summary}\NormalTok{(ulykker)}
\end{Highlighting}
\end{Shaded}

\begin{verbatim}
     agecat         gender        accid            pop        
 Min.   :1.00   Min.   :0.0   Min.   :54123   Min.   :187791  
 1st Qu.:1.25   1st Qu.:0.0   1st Qu.:57334   1st Qu.:196416  
 Median :2.00   Median :0.5   Median :60967   Median :199633  
 Mean   :2.00   Mean   :0.5   Mean   :60801   Mean   :199035  
 3rd Qu.:2.75   3rd Qu.:1.0   3rd Qu.:64610   3rd Qu.:202586  
 Max.   :3.00   Max.   :1.0   Max.   :66804   Max.   :208239  
\end{verbatim}

Datasetta er forøvrig
\href{https://www.ibm.com/docs/en/spss-statistics/saas?topic=tutorial-sample-files}{lånt
fra SPSS}. Du trenger ikke bruke disse datasetta i din egen gjennomgang,
bare finn to andre. Putt dem gjerne på forskjellige plasser for å øve på
å laste dem inn fra ulik lokasjon. Hvis du putter filstien i en vektor
slik jeg gjorde i det ene eksemplet, husk at du enten må putte begge
filstiene i ulike objekter, eller at du må laste inn den \emph{første}
fila før du overskriver filstia med den \emph{andre} filas filsti.

\begin{Shaded}
\begin{Highlighting}[]
\CommentTok{\# Dette funker}
\NormalTok{filsti\_atferd }\OtherTok{\textless{}{-}} \StringTok{"data/behavior.sav"}
\NormalTok{filsti\_ulykker }\OtherTok{\textless{}{-}} \StringTok{"C:}\SpecialCharTok{\textbackslash{}\textbackslash{}}\StringTok{Program Files}\SpecialCharTok{\textbackslash{}\textbackslash{}}\StringTok{IBM}\SpecialCharTok{\textbackslash{}\textbackslash{}}\StringTok{SPSS}\SpecialCharTok{\textbackslash{}\textbackslash{}}\StringTok{Statistics}\SpecialCharTok{\textbackslash{}\textbackslash{}}\StringTok{26}\SpecialCharTok{\textbackslash{}\textbackslash{}}\StringTok{Samples}\SpecialCharTok{\textbackslash{}\textbackslash{}}\StringTok{English}\SpecialCharTok{\textbackslash{}\textbackslash{}}\StringTok{accidents.sav"}

\NormalTok{atferd }\OtherTok{\textless{}{-}} \FunctionTok{read\_sav}\NormalTok{(}\AttributeTok{file =}\NormalTok{ filsti\_atferd)}
\NormalTok{atferd }\OtherTok{\textless{}{-}} \FunctionTok{read\_sav}\NormalTok{(}\AttributeTok{file =}\NormalTok{ filsti\_ulykker)}

\CommentTok{\# Dette funker også, men du må huske på å aldri endre rekkefølgen.}
\NormalTok{filsti }\OtherTok{\textless{}{-}} \StringTok{"data/behavior.sav"}
\NormalTok{atferd }\OtherTok{\textless{}{-}} \FunctionTok{read\_sav}\NormalTok{(}\AttributeTok{file =}\NormalTok{ filsti)}

\NormalTok{filsti }\OtherTok{\textless{}{-}} \StringTok{"C:}\SpecialCharTok{\textbackslash{}\textbackslash{}}\StringTok{Program Files}\SpecialCharTok{\textbackslash{}\textbackslash{}}\StringTok{IBM}\SpecialCharTok{\textbackslash{}\textbackslash{}}\StringTok{SPSS}\SpecialCharTok{\textbackslash{}\textbackslash{}}\StringTok{Statistics}\SpecialCharTok{\textbackslash{}\textbackslash{}}\StringTok{26}\SpecialCharTok{\textbackslash{}\textbackslash{}}\StringTok{Samples}\SpecialCharTok{\textbackslash{}\textbackslash{}}\StringTok{English}\SpecialCharTok{\textbackslash{}\textbackslash{}}\StringTok{accidents.sav"}
\NormalTok{ulykker }\OtherTok{\textless{}{-}} \FunctionTok{read\_sav}\NormalTok{(}\AttributeTok{file =}\NormalTok{ filsti)}

\CommentTok{\# Dette vil laste inn det samme datasettet to ganger, og dem ulike navn.}
\CommentTok{\# Begge vil være ulykker, hvis filsti blei definert sist.}
\NormalTok{filsti }\OtherTok{\textless{}{-}} \StringTok{"data/behavior.sav"}
\NormalTok{filsti }\OtherTok{\textless{}{-}} \StringTok{"C:}\SpecialCharTok{\textbackslash{}\textbackslash{}}\StringTok{Program Files}\SpecialCharTok{\textbackslash{}\textbackslash{}}\StringTok{IBM}\SpecialCharTok{\textbackslash{}\textbackslash{}}\StringTok{SPSS}\SpecialCharTok{\textbackslash{}\textbackslash{}}\StringTok{Statistics}\SpecialCharTok{\textbackslash{}\textbackslash{}}\StringTok{26}\SpecialCharTok{\textbackslash{}\textbackslash{}}\StringTok{Samples}\SpecialCharTok{\textbackslash{}\textbackslash{}}\StringTok{English}\SpecialCharTok{\textbackslash{}\textbackslash{}}\StringTok{accidents.sav"}

\NormalTok{atferd }\OtherTok{\textless{}{-}} \FunctionTok{read\_sav}\NormalTok{(}\AttributeTok{file =}\NormalTok{ filsti)}
\NormalTok{ulykker }\OtherTok{\textless{}{-}} \FunctionTok{read\_sav}\NormalTok{(}\AttributeTok{file =}\NormalTok{ filsti)}
\end{Highlighting}
\end{Shaded}




\end{document}
